\begin{itemize}
\itemsep1pt\parskip0pt\parsep0pt
\item
  \hyperref[LameAudio.spin-PublicFunctions]{Public Functions}
\item
  \hyperref[LameAudio.spin-PrivateFunctions]{Private Functions}
\end{itemize}

\hyperdef{}{LameAudio.spin-PublicFunctions}{\subsection{Public
Functions}\label{LameAudio.spin-PublicFunctions}}

\begin{itemize}
\itemsep1pt\parskip0pt\parsep0pt
\item
  \hyperref[LameAudio.spin-Start]{Start}
\item
  \hyperref[LameAudio.spin-SetADSR(attackvar,decayvar,sustainvar,releasevar)]{SetADSR(attackvar,
  decayvar, sustainvar, releasevar)}
\item
  \hyperref[LameAudio.spin-SetWaveform(waveformvar,volumevar)]{SetWaveform(waveformvar,
  volumevar)}
\item
  \hyperref[LameAudio.spin-LoadInstr(instrnum)]{LoadInstr(instrnum)}
\item
  \hyperref[LameAudio.spin-PlaySound(channel,note)]{PlaySound(channel,
  note)}
\item
  \hyperref[LameAudio.spin-StopSound(channel)]{StopSound(channel)}
\item
  \hyperref[LameAudio.spin-StopAllSound]{StopAllSound}
\item
  \hyperref[LameAudio.spin-PlaySequence(songAddrvar)]{PlaySequence(songAddrvar)}
\item
  \hyperref[LameAudio.spin-LoadSong(songBarAddrvar)]{LoadSong(songBarAddrvar)}
\item
  \hyperref[LameAudio.spin-PlaySong]{PlaySong}
\item
  \hyperref[LameAudio.spin-StopSong]{StopSong}
\item
  \hyperref[LameAudio.spin-SongPlaying]{SongPlaying}
\end{itemize}

\hyperdef{}{LameAudio.spin-Start}{\paragraph{Start}\label{LameAudio.spin-Start}}

\textbf{} ~Expand source

\lstset{style=spin}
\begin{lstlisting}
PUB Start
      
    parameter := @sine
    channelparam := (INITVAL << 8)
    channelADSR := LONG[@instruments][0]
    songchoice := SONGOFF
    looping := 0  
    play := 0
    
    
    repeat oscindexer from 0 to OSCREGS-1 step REGPEROSC
        oscRegister[oscindexer] := 0
        oscRegister[oscindexer+1] := 0
        oscRegister[oscindexer+2] := 0
        oscRegister[oscindexer+3] := 0
    oscindexer := 0
    oscindexcounter := 0
    
    cognew(@oscmodule, @parameter)    'start assembly cog
    cognew(LoopingSongParser, @LoopingPlayStack)
    
\end{lstlisting}

\hyperdef{}{LameAudio.spin-SetADSR(attackvar,decayvar,sustainvar,releasevar)}{\paragraph{SetADSR(attackvar,
decayvar, sustainvar,
releasevar)}\label{LameAudio.spin-SetADSR(attackvar,decayvar,sustainvar,releasevar)}}

\textbf{} ~Expand source

\lstset{style=spin}
\begin{lstlisting}
PUB SetADSR(attackvar, decayvar, sustainvar, releasevar)

    channelADSR := (channelADSR & A_MASK) + (attackvar << A_OFFSET)
    channelADSR := (channelADSR & D_MASK) + (decayvar << D_OFFSET)
    channelADSR := (channelADSR & S_MASK) + (sustainvar << S_OFFSET)
    channelADSR := (channelADSR & R_MASK) + (releasevar << R_OFFSET)
  
\end{lstlisting}

\hyperdef{}{LameAudio.spin-SetWaveform(waveformvar,volumevar)}{\paragraph{SetWaveform(waveformvar,
volumevar)}\label{LameAudio.spin-SetWaveform(waveformvar,volumevar)}}

\textbf{} ~Expand source

\lstset{style=spin}
\begin{lstlisting}
PUB SetWaveform(waveformvar, volumevar)

    channelADSR := (channelADSR & W_MASK) + (waveformvar << W_OFFSET)
    channelparam := (channelparam & $FFFF00FF) + (volumevar << 8)
\end{lstlisting}

\hyperdef{}{LameAudio.spin-LoadInstr(instrnum)}{\paragraph{LoadInstr(instrnum)}\label{LameAudio.spin-LoadInstr(instrnum)}}

\textbf{} ~Expand source

\lstset{style=spin}
\begin{lstlisting}
PUB LoadInstr(instrnum)

    channelADSR := LONG[@instruments][instrnum]
  
\end{lstlisting}

\hyperdef{}{LameAudio.spin-PlaySound(channel,note)}{\paragraph{PlaySound(channel,
note)}\label{LameAudio.spin-PlaySound(channel,note)}}

\textbf{} ~Expand source

\lstset{style=spin}
\begin{lstlisting}
PUB PlaySound(channel, note)

    if note < 128 and channel < VOICES
        oscindexer := channel << 2
        oscRegister[oscindexer] &= !KEYBITS          
        oscRegister[oscindexer+1] &= !ADSRBITS
        oscRegister[oscindexer] := note + KEYBITS
\end{lstlisting}

\hyperdef{}{LameAudio.spin-StopSound(channel)}{\paragraph{StopSound(channel)}\label{LameAudio.spin-StopSound(channel)}}

\textbf{} ~Expand source

\lstset{style=spin}
\begin{lstlisting}
PUB StopSound(channel)

    if channel < VOICES
        oscindexer := channel << 2          
        oscRegister[oscindexer] &= !KEYBITS 
\end{lstlisting}

\hyperdef{}{LameAudio.spin-StopAllSound}{\paragraph{StopAllSound}\label{LameAudio.spin-StopAllSound}}

\textbf{} ~Expand source

\lstset{style=spin}
\begin{lstlisting}
PUB StopAllSound

    repeat oscindexer from 0 to OSCREGS-1 step REGPEROSC
        oscRegister[oscindexer] &= !KEYBITS 
\end{lstlisting}

\hyperdef{}{LameAudio.spin-PlaySequence(songAddrvar)}{\paragraph{PlaySequence(songAddrvar)}\label{LameAudio.spin-PlaySequence(songAddrvar)}}

\textbf{} ~Expand source

\lstset{style=spin}
\begin{lstlisting}
PUB PlaySequence(songAddrvar)
    seqcursor := 0
    
    repeat while byte[songAddrvar][seqcursor] <> ENDOFSONG
        seqbyte := byte[songAddrvar][seqcursor]
        if seqbyte == TIMEWAIT
            repeatlong := byte[songAddrvar][seqcursor] << 13
            repeat repeatseqindex from 0 to repeatlong
            seqcursor += 2
        
        elseif seqbyte == NOTEON
            PlaySound(byte[songAddrvar][seqcursor+1],byte[songAddrvar][seqcursor+2])   
            seqcursor += 3
        
        elseif seqbyte == NOTEOFF
            StopSound(byte[songAddrvar][seqcursor+1])
            seqcursor += 2
\end{lstlisting}

\hyperdef{}{LameAudio.spin-LoadSong(songBarAddrvar)}{\paragraph{LoadSong(songBarAddrvar)}\label{LameAudio.spin-LoadSong(songBarAddrvar)}}

\textbf{} ~Expand source

\lstset{style=spin}
\begin{lstlisting}
PUB LoadSong(songBarAddrvar)

    barAddr := songBarAddrvar
    totalbars := byte[songBarAddrvar][0]
    repeatlong := byte[songBarAddrvar][1] << 8
    barres := byte[songBarAddrvar][2]
    loopsongPtr := barAddr + totalbars*(barres+BYTES_BARHEADER) + BYTES_SONGHEADER        
    
    songcursor := 0
    barcursor := 0
\end{lstlisting}

\hyperdef{}{LameAudio.spin-PlaySong}{\paragraph{PlaySong}\label{LameAudio.spin-PlaySong}}

\textbf{} ~Expand source

\lstset{style=spin}
\begin{lstlisting}
PUB PlaySong

    play := 1
\end{lstlisting}

\hyperdef{}{LameAudio.spin-StopSong}{\paragraph{StopSong}\label{LameAudio.spin-StopSong}}

\textbf{} ~Expand source

\lstset{style=spin}
\begin{lstlisting}
PUB StopSong

    play := 0
    StopAllSound
    
\end{lstlisting}

\hyperdef{}{LameAudio.spin-SongPlaying}{\paragraph{SongPlaying}\label{LameAudio.spin-SongPlaying}}

This function returns whether a song is currently playing

\textbf{} ~Expand source

\lstset{style=spin}
\begin{lstlisting}
PUB SongPlaying
    return play
        
\end{lstlisting}

\hyperdef{}{LameAudio.spin-PrivateFunctions}{\subsection{Private
Functions}\label{LameAudio.spin-PrivateFunctions}}

\begin{itemize}
\itemsep1pt\parskip0pt\parsep0pt
\item
  \hyperref[LameAudio.spin-FindLoopBarFromSongPointer]{FindLoopBarFromSongPointer}
\item
  \hyperref[LameAudio.spin-LoopingSongParser]{LoopingSongParser}
\end{itemize}

\hyperdef{}{LameAudio.spin-FindLoopBarFromSongPointer}{\paragraph{FindLoopBarFromSongPointer}\label{LameAudio.spin-FindLoopBarFromSongPointer}}

This function increments the loop pointer by the value of the song
pointr

\textbf{} ~Expand source

\lstset{style=spin}
\begin{lstlisting}
PRI FindLoopBarFromSongPointer

    barinc := 0
    barshift := 0
    repeat while barinc < byte[loopsongPtr][songcursor]
        barshift += barres+BYTES_BARHEADER
        barinc++
\end{lstlisting}

\hyperdef{}{LameAudio.spin-LoopingSongParser}{\paragraph{LoopingSongParser}\label{LameAudio.spin-LoopingSongParser}}

\textbf{} ~Expand source

\lstset{style=spin}
\begin{lstlisting}
PRI LoopingSongParser

    repeat

        if play == 1
            songcursor := 0
               
            ' iterate through song definition lines
            repeat while byte[loopsongPtr][songcursor] <> SONGOFF and play == 1  
                
                barcursor := songcursor
                
                
                ' iterate through loop definitions
                repeat linecursor from 0 to (barres-1)
                
                    songcursor := barcursor

                    ' play all notes defined in song definition
                    repeat while byte[loopsongPtr][songcursor] <> BAROFF and play == 1  
                        FindLoopBarFromSongPointer 
                        
                        bartmp := barshift+BYTES_SONGHEADER+BYTES_BARHEADER+linecursor
                        
                        if byte[barAddr][bartmp] == SNOP

                        elseif byte[barAddr][bartmp] == SOFF
                            StopSound( byte[barAddr][barshift+BYTES_SONGHEADER] )       
                            
                        else
                            PlaySound( byte[barAddr][barshift+BYTES_SONGHEADER] , byte[barAddr][bartmp] )  'channel, note

                            
                        songcursor += 1
                    
                    repeat repeatindex from 0 to repeatlong
                   
                songcursor += 1

            StopAllSound
\end{lstlisting}
