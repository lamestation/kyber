\section{Powering The Board}
\subsection{Parts Needed For This Section}

\begin{figure}[H]
		\centering
    \includegraphics[width = 2.5in]{attachments/9732135/9797698.jpg}\\
    \label{fig:zeldatwilight}
    \caption{\textcopyright  2006 Nintendo. \\ Note the luxurious resolution and detailed texture.}
\end{figure}

\begin{itemize}
\item
  1 x DC barrel jack
\item
  1 x Slide switch
\item
  1 x 5V regulator
\item
  1 x 3.3V regulator
\item
  3 x 100$\mu$F capacitors
\item
  1 x 4-pin socket
\item
  1 x Green LED
\item
  1 x 220$\Omega$ resistor
\item
  2 x Metric screw
\item
  2 x Metric nut
\end{itemize}

\subsection{Instructions}

The LameStation can be powered in one of two ways; using the on-board
4AAA battery pack or an external DC power supply. You do not need to
worry about conflicts between the two. The LS automatically switches
between the two sources using a shunt switch in the power jack. A
power-on LED is provided to prove, without a doubt, that the board is
actually on (you'll be happy to know, trust me).

The battery pack is a large, plastic obstruction to everything else you
need to assemble, so this is one of the last items to go on the board.
Instead, solder in the DC barrel jack for some unlimited power fun. Then
solder in the green LED and series resistor so that whenever you do
anything, you have a quick sanity check that the power has power at
least when troubleshooting.

Solder the DC barrel jack into \textbf{J1}. Do not solder in the
battery until later on. It's helpful for components like this to tape
or clip the component to the board before attempting to solder it.
That way it won't move around. I use electrical tape for this purpose.
You would also do well to get yourself a panoramic vise as it makes
soldering a million times easier.

Solder the large power switch into \textbf{SW1}.

Solder the 5V and 3.3V regulators into~\textbf{U2} and~\textbf{U3},
respectively. Bend them to be flat on the board. Do this before
installing the nearby capacitors or else it will be harder to get your
fingers around to bend it into place.

Solder 3 100$\mu$F capacitors into~\textbf{C3}, \textbf{C4}, and
\textbf{C5}. You'll want to leave a small amount of slack. These
capacitors are polarized, which means they have to be inserted in a
certain direction. You can determine the positive and negative
terminals by looking at the body of the capacitor. On the side of one
of the pins, you will find a strip of minus signs printed on the side.
This is the negative terminal. Now looking at the PCB, the capacitor's
footprint will have two holes, and one will have a plus sign. That
indicates the positive terminal, so make sure that the strip of minus
signs faces the other side, away from the `+' pad. Another pro tip: if
you have trouble getting the capacitor to stay on the board while
soldering, twist the leads around each other so that it will hold to
the board. You will just clip these off later, so you don't need to
worry about them touching so it's a nice fast way to get the component
to hold on.

Solder one of the 4-pin sockets into~\textbf{P4}. These pins provide
test points where you can
\href{Taking-Test-Measurements_9732196.html}{connect a voltmeter and
ammeter across the power supply} to take measurements.

Solder the green LED into~\textbf{D1} and a 220Ω resistor
into~\textbf{R24}. This will provide a power-on light to indicate when
the board is turned on, and is also an indicate of when there is a
short (if the power on but no light comes on).

This completes the power subsystem, except for the battery pack, which
will be connected later.

~

\begin{bclogo}[couleur=bgblue, arrondi =0 , logo=\bcbombe, barre=none,noborder=true]{Never blow up your LEDs again!}
\itshape The short lead wire on a capacitor or LED is the cathode, or negative terminal.
\end{bclogo}



Icon

\textbf{Watch That Green LED!}

The green LED does more than just tell you the board is on. It is also
an early warning system that you've done something horribly wrong. If
you've accidentally soldered something that has created a short in the
system, the green LED will not turn on, since no power can get to it.
\textbf{This may be the only warning you have}.

When you add a component, the first thing you should do when powering on
the board is watch that green LED. If it doesn't come on, \textbf{turn
off the power immediately}. This will reduce the chance of something
being fried on the board as a result of your mishap.

Icon

\textbf{Ground Connections}

You may have difficulty soldering connections that are wired to ground.
Just hold the iron to the pin longer until the solder flows.
