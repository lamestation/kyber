\section{Abstract}

From a very young age, computers have been an integral part of my life.  I first discovered programming in the form of Graal Action Script, a language for developing non-player characters in an online game I enjoyed.  From there, I discovered DarkBASIC 3D game programming, then later in college, EECS 10 forced me to become acquainted with the wonderful world of C/C++ programming.  I began to develop my own video game, which would come to be known as piXel.  The game's concept was nothing original, but the thrill of seeing my ideas begin to take shape in a real and tangible form that I could show anyone ignited a fierce motivation within me to keep working.  

Video game development requires an intimate understanding of a wide array of concepts from a multitude of fields; engineering, computer science, mathematics, signal processing, physics, just to name a few.  And the development of a game console dually requires a broad swath of skills and background knowledge to develop such a device, which nominally includes power distribution, serial/parallel communications, digital audio synthesis and processing, human interface, video signal generation, PCB design and fabrication, and even networking.  A tremendous amount of effort over many tiresome nights has been put into the production of just such a device, and while not necessarily on par with the products manufactured by today's million-dollar research and development teams, the LameStation quite sufficiently showcases the many intricate facets of the development of such a device, and has proven an incredibly valuable experience in furthering my understanding of computer and electrical engineering.

It is my hope that after viewing the breadth and extent of the work required, even the most cynical of readers may come away with a greater appreciation for the magnificient feats of science and engineering inherent in each and every gaming device, and understand that they are far more than toys.

\begin{flushright}--Brett Weir\end{flushright}
